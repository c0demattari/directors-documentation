\documentclass[12pt, unicode]{beamer}
\usetheme{Madrid}

\usepackage{luatexja}
\usepackage{url}
\usepackage{color}

\title{入部手続き}
\author{名古屋工業大学公認課外活動団体C0de}
\date{}

\begin{document}

\begin{frame}
    \maketitle
\end{frame}

\begin{frame}{内容}
    \begin{itemize}
        \item Slack
        \item 個人情報の提出
        \item 部費
        \item チュートリアル
        \item 部会・行事
        \item 部室・部則
    \end{itemize}
\end{frame}

\begin{frame}{Slack}
    \begin{block}{Slackとは?}
        \begin{itemize}
            \item 部で使っているコミュニケーションツール
            \item 部会や行事などの連絡が行われる
        \end{itemize}
    \end{block}

    \begin{block}{Slackのアカウント作成}
        \begin{enumerate}
            \item 普段使っているメールアドレスを教えてください
            \item 招待メールを送るので少しお待ちください
            \item 届いた招待メールの「今すぐ参加」をクリック
            \item スマホの場合はストアにてSlackアプリをダウンロード
            \item 氏名に自分のフルネームをローマ字で入力
            \item パスワードに任意のパスワードを入力
        \end{enumerate}
    \end{block}
\end{frame}

\begin{frame}{Slack}
    \begin{block}{チャンネルとは?}
        \begin{itemize}
            \item LINEでいうグループのこと
            \item パブリックチャンネルとプライベートチャンネルの二種類がある
        \end{itemize}
    \end{block}

    \begin{block}{始めから入っているチャンネルの紹介}
        \begin{itemize}
            \item \#general:OB/OGの方にも関係のあることについて話す
            \item \#active\_members:部会やプロジェクトについて話す
            \item \#questions:誰でも自由に質問することができる
            \item \#random:部の活動とは関係のないことについて話す
        \end{itemize}
    \end{block}

    \begin{center}
        {\large 重要な連絡を見逃さないように\\\textcolor{red}{PCとスマホにアプリをインストール}しておこう}
    \end{center}
\end{frame}

\begin{frame}{個人情報の提出}
    \begin{block}{入部するにあたって必要な個人情報}
        \begin{itemize}
            \item 氏名
            \item 学籍番号
            \item 現住所
            \item メールアドレス
        \end{itemize}
    \end{block}

    \begin{block}{個人情報の利用目的}
        \begin{itemize}
            \item 名古屋工業大学やゆうちょ銀行に提出するための名簿を作成するため
        \end{itemize}
    \end{block}

    \begin{center}
        {\large これらの個人情報を\\\textcolor{red}{Slackのダイレクトメッセージ}にて教えてください}
    \end{center}
\end{frame}

\begin{frame}{部費}
    \begin{block}{金額}
        \begin{itemize}
            \item 部員一人あたり年額1,000円
        \end{itemize}
    \end{block}

    \begin{block}{部費の利用目的}
        \begin{itemize}
            \item デバッグ用の端末や参考書などを購入するため
        \end{itemize}
    \end{block}
\end{frame}

\begin{frame}{部費}
    \begin{center}
        {\large \textcolor{red}{一ヶ月後までに1,000円}を\\C0deのゆうちょ銀行の口座に振り込んでください}
    \end{center}

    \begin{block}{ゆうちょ銀行から振り込む場合}
        \begin{itemize}
            \item 記号:12030
            \item 番号:15378031
        \end{itemize}
    \end{block}

    \begin{block}{ゆうちょ銀行以外から振り込む場合}
        \begin{itemize}
            \item 店名:208(ニゼロハチ)
            \item 預金種目:普通預金
            \item 口座番号:1537803
        \end{itemize}
    \end{block}
\end{frame}

\begin{frame}{チュートリアル}
    \begin{center}
        {\large \textcolor{red}{TechTrainというサービスのMISSION}を使います}
    \end{center}

    \begin{block}{TechTrainとは}
        \begin{itemize}
            \item 有名企業のエンジニアから実務が学べるオンラインコミュニティ
        \end{itemize}
    \end{block}

    \begin{block}{TechTrainのアカウント作成}
        \begin{enumerate}
            \item \textcolor{blue}{\url{https://techbowl.co.jp/techtrain}}にアクセス
            \item 「オンラインコミュニティに登録する」をクリック
            \item 画面の指示に従って必要事項を入力
        \end{enumerate}
    \end{block}
\end{frame}

\begin{frame}{チュートリアル}
    \begin{block}{チュートリアルの流れ}
        \begin{enumerate}
            \item \textcolor{red}{$めざせ!GitHubMaster$}というMISSIONでGitとGitHubについて勉強
            \item \textcolor{red}{$ポートフォリオサイト(自己紹介サイト)をつくってみよう$}というMISSIONでポートフォリオサイトを作成
            \item 自分が作成したポートフォリオサイトを7月の部会にて発表
            \item \textcolor{red}{$就活に便利!会社情報をみんなで共有しよう$}というMISSIONでチームでのモバイル/Webアプリ開発を体験
            \item 自分のチームが作成したモバイル/Webアプリを12月の部会にて発表
        \end{enumerate}
    \end{block}
\end{frame}

\begin{frame}{部会・行事}
    \begin{block}{部会}
        \begin{itemize}
            \item 毎月第二水曜日に開催
            \item 役員からの連絡やプロジェクトの進捗報告などが行われる
        \end{itemize}
    \end{block}

    \begin{block}{行事}
        \begin{itemize}
            \item 新歓BBQ(5月上旬)
            \item 夏合宿(9月中旬)
            \item 忘年会(12月下旬)
        \end{itemize}
    \end{block}
\end{frame}

\begin{frame}{部室・部則}
    \begin{block}{部室}
        \begin{itemize}
            \item 名古屋工業大学20号館2階202号室
            \item 部員であれば誰でも自由に利用することができる
        \end{itemize}
    \end{block}

    \begin{block}{部則}
        \begin{itemize}
            \item \textcolor{blue}{\url{https://github.com/c0demattari/terms-of-c0de}}にて公開している
            \item 興味があれば見てみてください
        \end{itemize}
    \end{block}
\end{frame}

\end{document}
